\documentclass{article}
\usepackage{amssymb}
\usepackage{amsmath}
\usepackage{centernot}
\DeclareMathOperator*{\Union}{\bigcup}

\begin{document}

\title{EE351k: Homework 2}
\author{Anthony Weems}
\date{\today}
\maketitle

% Problem 1
\subsection{}
\begin{enumerate}
    \item[a.] \( \mathbb{P}(two~kings,~one~ace) = \dfrac{\binom{4}{2} \binom{4}{1} 11!}{\binom{52}{13}} \)
    \item[b.] \( \mathbb{P}(one~ace|two~kings) = \dfrac{\binom{4}{2} \binom{4}{1} 11!}{\binom{4}{2}12!} \)
\end{enumerate}

% Problem 2
\subsection{}
Counting problem:

\( \#~of~1~color~only = 3 \)

\( \#~of~2~colors = 3*2^k \)

\( \#~of~total~combinations = 3^k \)

\( \mathbb{P}(not~all~colors) = \dfrac{3+3*2^k}{3^k} \)

% Problem 3
\subsection{}

\begin{enumerate}
    \item[a.] \( \mathbb{P} = \dfrac{\binom{15}{6}}{\binom{25+15+35}{6}} \)
    \item[b.] \( \mathbb{P} = \dfrac{\binom{25}{2} \binom{15}{3} \binom{35}{1}}{\binom{25+15+35}{6}} \)
\end{enumerate}

% Problem 4
\subsection{}
Conditional Probability: \( \mathbb{P}(A|B) = \dfrac{\mathbb{P}(B|A)\mathbb{P}(A)}{\mathbb{P}(B|A)\mathbb{P}(A) + \mathbb{P}(B|A^c)\mathbb{P}(A^c)} \)

\( \mathbb{P}(D|T) = \dfrac{0.98*0.6}{0.98*0.6 + 0.02*0.4} = 0.9866 \)

% Problem 5
\subsection{}

\( \mathbb{P}(Four~heads \cap Dice_4) = \mathbb{P}(Four~heads)\mathbb{P}(Dice_4) = \dfrac{1}{96} \)

% Problem 6
\subsection{}
\begin{align*}
\mathbb{P}(C | A \cap B) &= \mathbb{P}(C | B) \\
\dfrac{\mathbb{P}(C \cap A \cap B)}{\mathbb{P}(A \cap B)} &= \dfrac{\mathbb{P}(C \cap B)}{\mathbb{P}(B)}\\
\dfrac{\mathbb{P}(C \cap A \cap B)}{\mathbb{P}(C \cap B)} &= \dfrac{\mathbb{P}(A \cap B)}{\mathbb{P}(B)}\\
\mathbb{P}(A | B \cap C) &= \mathbb{P}(A | B)
\end{align*}

% Problem 7
\subsection{}
Proof by induction:

\( n = 0 \implies \mathbb{P} = \frac{1}{2} [1+\frac{2}{3}^0] = 1 \)
\\
Assuming n case, attempt n+1 case \\
If the previous n were even, the probability that n+1 is also even...

\( \mathbb{P} = \frac{5}{6} \frac{1}{2} [1+\frac{2}{3}^n] = \frac{1}{2} [1+\frac{2}{3}^{n+1}] \)
\\
If the previous n were odd, the probability that n+1 is even...

\( \mathbb{P} = \frac{1}{6} (1 - \frac{1}{2} [1+\frac{2}{3}^n]) = \frac{1}{2} [1+\frac{2}{3}^{n+1}] \)

% Problem 8
\subsection{}

\begin{enumerate}
    \item[a.] \( \frac{1}{2}^n \)
    \item[b.] \( \binom{n}{n/2} \frac{1}{2^n} \)
    \item[c.] \( \binom{n}{2} \frac{1}{2^n} \)
    \item[d.] \( \sum\limits_{i=2}^n \binom{n}{i} \frac{1}{2^n} \)
\end{enumerate}


\end{document}
